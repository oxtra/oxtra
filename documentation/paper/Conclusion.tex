% Oxtra ist geil und ist geil und fokusiert auf zwei architekturen
Oxtra's focus on two specific architectures enables it to outperform the leading competitor QEMU in many cases. 
In benchmarks performed against a QEMU emulation, oxtra is up to 20\% faster than QEMU when executing the command line tool gzip and generally outperforms QEMU when executing low to medium complexity programs. 
Notably, as the comparison is based on a QEMU emulation, the results might be different on real RISC-V hardware. 

For a rough indicator of performance on actual hardware, we also analyzed the number of generated instructions and actually executed instructions (including the instructions required for code translation). 
In our test suite, oxtra generates about 20\% fewer instructions than QEMU and executes about 10--40\% fewer instructions than QEMU.
With non-trivial input, the proportional advantage of oxtra over QEMU decreases but consistently remains above 10\%.
\\

\noindent Oxtra currently only supports a small subset of x86-64 instructions. 
Most notable is the lack of support for SSE instructions. While adding specific instructions of an already supported instruction set is straightforward, adding a completely new instruction set (extension) potentially requires modifying the core architecture.

As of now, oxtra is only able to translate statically linked Linux programs and is incapable of loading libraries at runtime.
This is prohibited by the internal structure of the code store. 
Additionally, multi-threaded programs and self-modifying code are not supported.

In oxtra, code can only be translated once.
This implies that instructions that logically belong to the same block but are translated later must be connected through static links. 
Allowing code to be translated again (replacing previous blocks) would solve this issue and is the next step for improving performance.
Flag evaluation can also be improved: Currently, the instruction that updates the flag stores information to memory and the instruction that uses the flag loads this information from memory before calculating the state. 
This can be avoided by evaluating the flag directly and is the more common case. 
\\

% was wären die nächsten logischen steps für oxtra (instruktionen hinzufügen etc.)
% also generell die zukunft
\noindent The current version of oxtra is limited in its capabilities but still proves to have a lot of potential and can be used as a starting point for a performant RISC-V binary translator.